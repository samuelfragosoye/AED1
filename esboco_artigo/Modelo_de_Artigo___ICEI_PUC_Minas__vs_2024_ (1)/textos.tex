%%%%%%%%%%%%%%%%%%%%%%%%%%%%%%%%%%%%%%%%%%%%%%%%%%%%%%%%%%%%%%%%%%%%%%%%%%%%%%%%%%%%%%%%%%%%%%%%%%%%%%%%%%%%%%%%%%%%%%%%%%%%%%%%
%%%%%%%%%%%%%% Template de Artigo Adaptado para Trabalho de Conclusão de Curso - SI Contagem - PUCMINAS                       %%
%% codificação UTF-8 - Abntex - Latex -  							                                                          %%
%% Autor da primeira versão:    Fábio Leandro Rodrigues Cordeiro                                                              %% 
%% Co-autores da primeira versão: Prof. João Paulo Domingos Silva, Harison da Silva e Anderson Carvalho		                  %%
%% Revisores normas NBR (Padrão PUC Minas) da primeira versão: Helenice Rego Cunha e Prof. Theldo Cruz                        %%
%% Versão: 1.1     18 de dezembro 2015                                                                                        %%
%%%%%%%%%%%%%%%%%%%%%%%%%%%%%%%%%%%%%%%%%%%%%%%%%%%%%%%%%%%%%%%%%%%%%%%%%%%%%%%%%%%%%%%%%%%%%%%%%%%%%%%%%%%%%%%%%%%%%%%%%%%%%%%%
\section{\esp Introdução (1 página)} 

O presente trabalho tem como objetivo apresentar uma síntese das classificações de pesquisa científica abordadas por Gilson Volpato\cite{volpato2012}. Será feita a distinção entre os tipos lógicos de pesquisa, com foco na diferenciação entre pesquisas que necessitam da criação de hipóteses - como no caso da associação e interferência - e das que não possuem essa exigência para atingir o resultado.


\section{\esp Referencial Teórico (1 página)}
Deixando de lado as especifidades e qualificações de uma pesquisa e entrando em sua raíz lógica podemos defini-la em até três tipos. Entender esses três tipos é fundamental para todo o processo de fazer uma pesquisa e escrever um artigo. A pesquisa se inicia com uma pergunta sobre um fato observado, essa resposta pode ser, em dois dos três tipos, respondida com o auxílio de hipóteses científicas. De acordo com a Encyclopaedia Britannica (2024)\cite{britannica2024}, uma hipótese científica é uma explicação inicial para um fato observado, que ainda precisa ser testada. No primeiro caso a pergunta não precisa necessariamente de uma hipótese para ser respondida como no exemplo da pergunta "Que rotas de migração são utilizadas por uma população de baleias jubarte?". Aqui o objetivo é rastrear e descrever o deslocamento, é uma questão de "o quê" e "onde", não "por quê" e por isso não se faz necessário o uso de uma hipótese. A coleta de dados é direcionada apenas pela pergunta e não por uma hipótese que nesse caso não passaria de um capricho acadêmico.  A hipótese se faz necessária quando a pergunta da pesquisa busca explicar ou comparar. No exemplo da pergunta "Por que a luz não acende?" propõe-se a hipótese de que "A lâmpada queimou", depois de testá-la em um soquede e descobrir que estava errada, formula-se uma nova hipótese: "interruptor quebrado". Com a realização de novos testes, a hipótese é confirmada. Nesse exemplo a coleta de dados foi direcionada pela hipótese e a elaboração dela foi crucial para a descoberta do resultado, mostrando-se uma ferramenta fundamental para chegar na resposta do problema. É importante que as primeiras hipóteses sejam as mais simples e se forem refutadas inicia-se os testes de hipótese mais complexas. As pesquisas com hipóteses examinam a correlação ou causalidade entre variáveis e são dividas em dois tipos: as de associação e interferência. A associação é uma relação onde duas variáveis mudam de forma previsível, mas sem uma relação de causa e efeito direta. Como é mostrado no artigo "New evidence for the Theory of the Stork"\cite{hofer2004}, de Thomas Höfer, onde cita dados de Berlim que mostram uma correlação entre crescimento do número de cegonhas (variável A) na cidade com partos fora de hospitais (variável B) mas não há qualquer relação entre hospitais e população de cegonhas, mas pode haver uma variável C oculta influenciando ambas, caracterizando uma relação sem causa e efeito entre elas. A interferência ocorre quando há uma relação de causa, mecanismo e efeito onde as variáveis estão relacionadas e existe uma relação de causa e efeito sobre elas. Portanto, os três tipos lógicos de pesquisa são: descrição, associação e associação com interferência.




\section{\esp Conclusão}


Este trabalho apresentou a distinção fundamental entre os três tipos lógicos de pesquisa: descritiva, de associação e de interferência. Foi demonstrado que a principal diferença entre eles reside na necessidade e no uso de hipóteses para guiar a investigação. A compreensão dessa estrutura é essencial para a elaboração de projetos de pesquisa claros e metodologicamente corretos, garantindo que as perguntas, os métodos e as conclusões de um estudo científico estejam alinhados de forma lógica e coerente.


